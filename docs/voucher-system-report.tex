\documentclass[12pt,a4paper]{report}
\usepackage{fontspec}
\usepackage{polyglossia}
\setmainlanguage{vietnamese}
\setmainfont{Times New Roman}
\usepackage{geometry}
\geometry{margin=2.5cm}
\usepackage{setspace}
\onehalfspacing
\usepackage{graphicx}
\usepackage{hyperref}
\hypersetup{colorlinks=true, linkcolor=blue, urlcolor=blue}
\usepackage{array}

\begin{document}

\begin{titlepage}
    \centering
    {\Large\textbf{BÁO CÁO KIẾN TRÚC \& ĐẶC TẢ YÊU CẦU}}\\[1cm]
    {\huge\textbf{Voucher Management System}}\\[1cm]
    {\large Node.js -- Express -- MongoDB -- EJS}\\[2cm]
    {\large Đồ án chuyên ngành}\\
    {\large Trường Đại học ...}\\[2cm]
    \begin{flushright}
        \textbf{Sinh viên thực hiện:}\\
        Nguyễn Văn A -- MSSV 20xx\\
        Lớp: Dxx\\[0.5cm]
        \textbf{Giảng viên hướng dẫn:}\\
        ThS. Trần Văn B
    \end{flushright}
    \vfill
    {\large TP. HCM, 11/2025}
\end{titlepage}

\tableofcontents
\newpage

\chapter*{Lời mở đầu}
Trong bối cảnh chuyển đổi số mạnh mẽ của ngành dịch vụ ăn uống và giải trí, nhu cầu quản lý các chương trình khuyến mãi và chăm sóc khách hàng trở nên cấp thiết. Dự án \textit{Voucher Management System} được xây dựng nhằm tạo ra một nền tảng tập trung cho phép người dùng cuối khám phá địa điểm và nhận ưu đãi nhanh chóng, đồng thời hỗ trợ chủ địa điểm cũng như quản trị viên vận hành và kiểm soát hệ thống. Báo cáo này mô tả toàn bộ quá trình khảo sát, phân tích, thiết kế và đặc tả yêu cầu của hệ thống theo chuẩn tài liệu kỹ nghệ phần mềm.

\addcontentsline{toc}{chapter}{Lời mở đầu}

\chapter{Cơ sở lý thuyết}

\section{Bản chất của phần mềm}
\subsection{Định nghĩa phần mềm}
Trong phạm vi đồ án, phần mềm được hiểu là tập hợp chương trình, dữ liệu và tài liệu liên quan nhằm cung cấp dịch vụ quản lý voucher, bao gồm ứng dụng web (Express + EJS), cơ sở dữ liệu MongoDB và các script vận hành.

\subsection{Các lĩnh vực ứng dụng của phần mềm}
Giải pháp tập trung vào lĩnh vực F\&B, du lịch và bán lẻ địa phương. Các use case chính:
\begin{itemize}
    \item B2C: người dùng tìm kiếm địa điểm và ưu đãi.
    \item B2B2C: chủ địa điểm quản trị voucher, theo dõi phản hồi.
    \item Nội bộ: admin giám sát, đảm bảo tuân thủ chính sách.
\end{itemize}

\subsection{Phần mềm kế thừa}
Hệ thống kế thừa một số tài nguyên: dữ liệu địa điểm từ các chiến dịch marketing trước, module upload media từ dự án review nội bộ. Việc tái sử dụng giúp giảm thời gian phát triển nhưng yêu cầu chuẩn hóa lại kiến trúc và quy trình kiểm thử.

\section{Kỹ nghệ phần mềm}
\subsection{Định nghĩa và tầm quan trọng}
Kỹ nghệ phần mềm là việc áp dụng có hệ thống các nguyên lý kỹ thuật vào phát triển, vận hành và bảo trì phần mềm. Với dự án voucher, kỹ nghệ phần mềm giúp đảm bảo chất lượng (bằng chứng audit), rút ngắn thời gian triển khai và dễ dàng mở rộng.

\subsection{Các lớp và thành phần}
Áp dụng mô hình 4 lớp: \textit{Process} (Scrum 2 tuần), \textit{Methods} (phân tích use case, ERD, sequence), \textit{Tools} (Node.js, MongoDB, Git), \textit{Quality focus} (lint, seed test data, checklist deploy).

\subsection{Nguyên tắc thực hành và hướng tiếp cận}
\begin{itemize}
    \item \textbf{Iterative delivery}: mỗi sprint phát hành một phiên bản demo.
    \item \textbf{Validation-first}: build seed data \& script kiểm thử.
    \item \textbf{Documentation-as-code}: tài liệu như file XeLaTeX trong repo, versioned cùng source.
\end{itemize}

\section{Cấu trúc quy trình phần mềm}
\subsection{Mô hình quá trình chung}
Quy trình dự án gồm các pha: khảo sát hiện trạng $\rightarrow$ phân tích yêu cầu $\rightarrow$ thiết kế kiến trúc $\rightarrow$ triển khai $\rightarrow$ kiểm thử $\rightarrow$ triển khai thực tế $\rightarrow$ vận hành.

\subsection{Đánh giá và cải tiến}
Sau mỗi sprint, nhóm đánh giá velocity, chất lượng release (số bug), từ đó tinh chỉnh backlog và automation (ví dụ thêm script seed, lược bỏ module NLP khi không đáp ứng KPI).

\section{Các mô hình quy trình phát triển phần mềm}
Do đặc thù cần release sớm cho nhà hàng thử nghiệm, nhóm chọn \textbf{mô hình phát triển tăng dần} kết hợp nguyên tắc \textbf{V-model} cho test. Các mô hình khác được phân tích trong phụ lục: thác nước, V, tiến hóa, nguyên mẫu, xoắn ốc, song song.

\section{Khảo sát hệ thống}
\subsection{Khảo sát hiện trạng tổ chức}
Phỏng vấn các chủ quán và phòng marketing cho thấy việc quản lý voucher hiện dựa trên bảng tính và mạng xã hội, thiếu phân quyền và báo cáo tập trung.

\subsection{Khảo sát nghiệp vụ}
Nghiệp vụ chính:
\begin{itemize}
    \item Tạo voucher theo chiến dịch, giới hạn số lượng, thời gian.
    \item User đến địa điểm, xuất trình mã, chủ quán cần xác thực.
    \item Admin phải có khả năng thu hồi voucher vi phạm.
\end{itemize}
Sơ đồ quy trình hiện tại cho thấy nhiều bước thủ công (Google Sheet, Messenger).

\subsection{Khảo sát hiện trạng tin học}
Hạ tầng gồm máy tính văn phòng, kết nối internet ổn định; chủ quán thường dùng laptop hoặc tablet. Không có hệ thống quản lý tập trung.

\section{Phân tích yêu cầu}
\subsection{Khái niệm và loại yêu cầu}
Yêu cầu chức năng bao gồm: quản lý user/owner/admin, CRUD địa điểm/voucher, claim voucher, review có media, dashboard thống kê. Yêu cầu phi chức năng: phản hồi $<$ 2s, hỗ trợ 500 user đồng thời (theo dự báo), bảo mật session, backup.

\subsection{Phương pháp thu thập yêu cầu}
Sử dụng bảng hỏi Google Form cho người dùng cuối, phỏng vấn bán cấu trúc với chủ quán, quan sát trực tiếp quy trình claim hiện tại và phân tích tài liệu marketing cũ.

\subsection{Mô hình hóa yêu cầu}
Use case tổng quan đã trình bày trong phần 2.2; ngoài ra hệ thống còn có sequence diagram cho login, claim voucher; activity diagram mô tả luồng review.

\section{Đặc tả yêu cầu}
\subsection{Phương pháp đặc tả}
Áp dụng chuẩn IEEE 830 / ISO/IEC/IEEE 29148 cho tài liệu SRS: mô tả phạm vi, bối cảnh, yêu cầu chức năng/phi chức năng, ràng buộc, giả định.

\subsection{Vai trò và đặc điểm SRS}
SRS là cơ sở thống nhất giữa nhóm dev, QA và khách hàng (nhà hàng). Tài liệu được lưu trong Git để kiểm soát version, review qua Pull Request.

\subsection{Thành phần chính của SRS}
\begin{itemize}
    \item Introdution (mục tiêu, phạm vi, định nghĩa).
    \item Overall description (perspective, user class, constraint).
    \item Specific requirements (functional, interface, performance, security).
    \item Appendices (glossary, seed data, mockup).
\end{itemize}

\newpage

\chapter{Khảo sát hiện trạng hệ thống}
\section{Khảo sát tổ chức}
\subsection{Cơ cấu tổ chức}
Doanh nghiệp mục tiêu gồm phòng marketing, phòng vận hành (owner relation), phòng CNTT. Admin của hệ thống thuộc phòng CNTT; owner là đối tác bên ngoài.

\subsection{Sơ đồ tổ chức \& chức năng}
Sơ đồ minh họa các phòng ban và mối quan hệ được trình bày trong phụ lục A. Mỗi phòng chịu trách nhiệm cung cấp dữ liệu và kiểm thử chức năng liên quan.

\section{Khảo sát nghiệp vụ}
\subsection{Nghiệp vụ chính}
Liệt kê chi tiết từng nghiệp vụ: tiếp nhận đề xuất địa điểm, phê duyệt, phát hành voucher, kiểm soát claim, thu thập review, báo cáo doanh thu từ voucher.

\subsection{Bảng mô tả nghiệp vụ}
Sử dụng bảng gồm các cột: Mã nghiệp vụ, Mô tả, Đầu vào, Đầu ra, Actor. (Bảng cụ thể nằm trong phụ lục B).

\subsection{Sơ đồ quy trình hiện tại}
Activity diagram mô tả quy trình claim hiện tại (truyền thống) vs. quy trình mới. Sơ đồ được dựng bằng Mermaid và đính kèm trong phụ lục C.

\section{Khảo sát hiện trạng tin học}
\subsection{Nhân sự \& trình độ}
Nhân viên nội bộ sử dụng thành thạo ứng dụng web; chủ quán quen với giao diện mobile/responsive.

\subsection{Hạ tầng mạng \& thiết bị}
Máy chủ dự kiến đặt tại VPS; các phòng ban có mạng LAN nội bộ. Chủ quán sử dụng thiết bị cá nhân nên yêu cầu giao diện responsive và ít phụ thuộc plugin.

\subsection{Phần mềm hiện có}
Đang dùng Google Workspace, Trello cho quản lý chiến dịch; chưa có hệ thống CRM/Voucher chuyên dụng.

\chapter{Phân tích yêu cầu}
\section{Phương pháp nắm bắt yêu cầu}
\subsection{Bảng hỏi (Google Form)}
Bảng hỏi gồm 15 câu nhằm xác định hành vi người dùng khi nhận ưu đãi. Kết quả: 78\% mong muốn xem review kèm hình ảnh, 65\% muốn lưu lịch sử voucher.

\subsection{Quan sát quy trình}
Nhóm thực tế tại 3 quán cà phê: ghi nhận thời gian xử lý voucher hiện tại trung bình 2 phút/voucher, nhiều lỗi ghi nhận do thiếu kiểm tra số lượng còn lại.

\section{Kết quả phân tích}
\subsection{Danh sách yêu cầu chức năng}
\begin{itemize}
    \item F1: User đăng ký/đăng nhập, cập nhật hồ sơ.
    \item F2: User duyệt địa điểm, xem chi tiết, claim voucher.
    \item F3: Owner CRUD địa điểm/voucher thuộc sở hữu.
    \item F4: Admin quản lý user/location/voucher/review, xuất báo cáo.
    \item F5: Hệ thống cho phép review có media và xử lý upload.
\end{itemize}

\subsection{Yêu cầu phi chức năng}
\begin{itemize}
    \item NFR1: Thời gian phản hồi trang chủ \textless{} 2 giây với 200 người dùng đồng thời.
    \item NFR2: Tuân thủ RBAC, session timeout 24h, hỗ trợ HTTPS.
    \item NFR3: Backup MongoDB và thư mục uploads hằng ngày.
\end{itemize}

\subsection{Bảng tổng hợp yêu cầu}
Bảng RACI (trách nhiệm) được xây dựng để phân tích ai chịu trách nhiệm cho từng yêu cầu, giúp liên kết với sprint backlog.

\subsection{Use case tổng quan \& mô tả}
Sơ đồ use case tổng quan nằm trong chương 2; các use case chính (claim voucher, quản lý voucher, kiểm duyệt review) được mô tả theo template: Mục tiêu, Actor, Preconditions, Main flow, Alternative flow, Exception.

\chapter{Đặc tả yêu cầu}
\section{Phương pháp đặc tả}
Áp dụng \textbf{SRS-based} kết hợp \textbf{User Story} cho backlog. Các user story được ghi trong Jira/Notion, liên kết với test case.

\section{Tài liệu đặc tả yêu cầu}
Tài liệu SRS chi tiết bao gồm:
\begin{itemize}
    \item \textbf{Functional specification}: mô tả từng API, view, quyền truy cập.
    \item \textbf{Data dictionary}: định nghĩa schema, kiểu dữ liệu, validation.
    \item \textbf{Interface specification}: mockup EJS, layout, thông điệp lỗi.
    \item \textbf{Operational manual}: hướng dẫn seed dữ liệu, deploy (tham chiếu `README.md`, `SETUP.md`).
\end{itemize}

---

\section*{Kết luận}
Hệ thống Voucher Management System được thiết kế theo hướng monolith nhưng đã chuẩn hoá quy trình phát triển, tài liệu và deployment. Việc áp dụng đầy đủ các bước khảo sát, phân tích và đặc tả giúp đảm bảo sản phẩm đáp ứng yêu cầu nghiệp vụ, dễ dàng mở rộng và chuyển giao trong tương lai.

\addcontentsline{toc}{section}{Kết luận}

\end{document}

